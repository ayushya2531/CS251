\documentclass{article}

\usepackage{graphicx}
\usepackage{float}

\begin{document}

\section*{Scatter Plot 1}

\begin{figure}[H]
  \centering
  \includegraphics[width=\linewidth]{dataplot_1.eps}
  \caption{Threads = 1}
\end{figure}

Scatter plot when executable is run with 1 thread
\newline
As can be seen, the time increases with the number of elements, which was supposed to be the case
\newline

\section*{Scatter Plot 2}

\begin{figure}[H]
  \centering
  \includegraphics[width=\linewidth]{dataplot_2.eps}
  \caption{Threads = 2}
\end{figure}

Scatter plot when executable is run with 2 threads
\newline
As can be seen, the time increases with the number of elements, which was supposed to be the case
\newline
\section*{Scatter Plot 3}

\begin{figure}[H]
  \centering
  \includegraphics[width=\linewidth]{dataplot_4.eps}
  \caption{Threads = 4}
\end{figure}

Scatter plot when executable is run with 4 threads
\newline
As can be seen, the time increases with the number of elements, which was supposed to be the case
\newline
\section*{Scatter Plot 4}

\begin{figure}[H]
  \centering
  \includegraphics[width=\linewidth]{dataplot_8.eps}
  \caption{Threads = 8}
\end{figure}

Scatter plot when executable is run with 8 threads
\newline
As can be seen, the time increases with the number of elements, which was supposed to be the case
\newline
\section*{Scatter Plot 5}

\begin{figure}[H]
  \centering
  \includegraphics[width=\linewidth]{dataplot_16.eps}
  \caption{Threads = 16}
\end{figure}

Scatter plot when executable is run with 16 threads
\newline
As can be seen, the time increases with the number of elements, which was supposed to be the case
\newline

\section*{Line Plot}

\begin{figure}[H]
  \centering
  \includegraphics[width=\linewidth]{line_plot.eps}
  \caption{Line Plot}
\end{figure}

\section*{Bar Plot}

\begin{figure}[H]
  \centering
  \includegraphics[width=\linewidth]{bar_plot.eps}
  \caption{Bar Plot}
\end{figure}

\end{document}
