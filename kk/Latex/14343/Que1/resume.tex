\documentclass[a4paper,10pt]{article}

\usepackage{geometry}
\geometry{
    a4paper,
    left=0.75in,
    top=0.75in,
    right=0.75in,
    bottom=0.75in
}

\begin{document}
\noindent
\textbf{Kunal Kapila}\\
Department of Mathematics and Statistics\\
Double Major, Computer Science and Engineering\\
Indian Institute of Technology Kanpur, India\\
\textit{Email: kunalkap@iitk.ac.in}\\

\hrule width \hsize \kern 1mm \hrule width \hsize \kern 2mm
\noindent
\textit{This paragraph is about myself}. I am interested in Programming and Theoretic Mathematics. I'm not sure what I want to do in life yet. I will stop here.
\vspace{2mm}
\hrule width \hsize \kern 1mm \hrule width \hsize

\subsubsection*{Education}
\begin{tabular}{ | l | c | l | l | }
    \hline
    Class 10 & 2012 & Delhi Public School, R. K. Puram & 10 \\
    \hline
    Class 12 & 2014 & Delhi Public School, R. K. Puram & 98.2\% \\
    \hline
    Bachelor of Science & 2014 - 2019 & Indian Institute of Technology, India & 8.7\\
    \hline
\end{tabular}
\subsubsection*{Course details}
\begin{tabular}{ |l | l | l |  }
    \hline
    \textbf{Semester} & \textbf{Total credits} & \textbf{Highlights}\\
    \hline
    July 2016 - November 2016 & 45 & Algorithms 2 \\
    \hline
    January 2017 - April 2017 & 63 & Computational Number Theory and Algebra\\
    \hline
    July 2017 - November 2017 & 49 & Theory of Computation\\
    \hline
\end{tabular}
\subsubsection*{Skills}
\begin{tabular}{ l  l  l   }
    Computer Science & : & Data Structures and Algorithms, Theory of Computation \\
    Programming languages & : & C, C++, Python, Bash, Awk \\
    Tools & : & Octave, Docker, Git, \LaTeX \\
\end{tabular}
\subsubsection*{Projects}
\begin{enumerate}
    \item \textbf{Cimulator}: a Visual \& Interactive Tutoring system - Interpreter for C language in python, modelled memory artificially, handled external header files \& overflows, Prompts user with possible corrections for runtime errors instead of crashing abruptly (unlike gcc), Web interface to \textbf{simulate C codes} that were interpreted by Cimulator to provide visual cues to the user
    \item Head Web, Antaragni 2016, Students' Gymkhana - Conceptualised and designed the website for Antaragni 2016. Implemented server using MEAN stack along with \textit{ejs} for templating \& \textit{passport} for facebook authentication
\end{enumerate}

\subsubsection*{Awards and achievements}

\renewcommand{\labelitemi}{$\textendash$}
\begin{itemize}
    \item Secured AIR 116 in \textbf{JEE Mains} 2014 \& AIR 1306 in \textbf{JEE Advanced} 2014 amongst 1.5 million candidates
    \item Secured Rank 29 \& 54 in \textbf{ACM-ICPC Amritapuri} Onsite \& Online Round 2015 respectively amongst 1500 teams
    \vspace{-5pt}
\end{itemize}

\subsubsection*{Hobbies}
\renewcommand{\labelitemi}{$\textendash$}
\begin{itemize}
    \item Policy Making
    \item Watching Sports: Cricket, Tennis
    \item Watching Movies, TV Series
    \item Listening to Music
\end{itemize}
\end{document}
