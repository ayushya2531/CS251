\documentclass{article}

\usepackage{geometry}
\geometry{
 a4paper,
 total={170mm,257mm},
 left=20mm,
 top=10mm,
 right=20mm,
 bottom=20mm
 }

\begin{document}
\noindent
\textbf{Ayushya Agarwal}\\
Department of Electrical Engineering\\
Indian Institute of Technology Kanpur, India\\
\textit{Email: ayushya@iitk.ac.in}\\

\hrule width \hsize \kern 1mm \hrule width \hsize 
\vspace{2mm}
I'm a senior undergrad in the Department of Electrical Engineerin at the Indian Institute of Technology Kanpur. I am interested in Artificial Intelligence, Machine Learning, Signal Processing and Bio-mimicry. Apart from my academic interests, I moonlight as a crusader for correcting systematic flaws through the Students' Senate, am an avid reader of fantasy novels and an occasional poet.
\vspace{2.5mm}
\hrule width \hsize \kern 1mm \hrule width \hsize

\subsubsection*{Education}

\begin{tabular}{|l|l|l|l|}
\hline
Matriculation & 2012  & City Montessori School, Lucknow        & 97.8\% \\ \hline
Intermediate  & 2014  & Aklank Public School, Kota             & 93.6\% \\ \hline
B.Tech.       & 2018* & Indian Institute of Technology, Kanpur & 8.8    \\ \hline
\end{tabular}
\subsubsection*{Course details}
\begin{tabular}{ |l | l | l |  }
    \hline
    \textbf{Semester} & \textbf{Total credits} & \textbf{Highlights}\\
    \hline
    July 2016 - November 2016 & 49 & Machine Learning Theory and Digital Electronics \\
    \hline
    January 2017 - April 2017 & 63 & Computer Organisation and Neural Networks\\
    \hline
    July 2017 - November 2017 & 57 & Operating Systems, Visual Recognition and Probabilistic ML\\
    \hline
\end{tabular}
\subsubsection*{Skills}
\begin{tabular}{ l  l  l   }
    Programming languages & : & C, C++, Python, Bash, Awk, R, MATLAB, Verilog \\
    Frameworks and Libraries & : & Tensorflow, Pytorch, Keras, Matplotlib, Pandas, Scikit-learn, OpenCV \\
    Tools & : & Git, \LaTeX, Vim \\
\end{tabular}
\subsubsection*{Projects}
\begin{enumerate}
    \item Learning Methods for Nearfield Beamforming - Supervisor: Prof. Rajesh Hegde, Department of EE, IIT Kanpur. The project was based on improving upon existing range estimation techniques through deep learning based approaches to benefit near-field beamforming. I analyzed the usage of Spherical Harmonic features in a learning framework for range estimation of signal source.
    \item Human Pose Estimation and Segmentation - Supervisor: Prof. Vinay Namboodri, Department of CSE, IIT Kanpur. The aim was to develop an end to end fully convolutional model for joint human pose point estimation  and human part segmentation via hard parameter sharing.I created a flexible pipeline for joint learning of the two complementary tasks, inspired by the CVPR'17 paper on pose estimation using Part Affinity Fieldss and the standard FCN architecture
    \item Multi-task learning using Bayesian SVMs - Supervisor : Prof. Piyush Rai, Department of CSE, IIT Kanpur. The aim was to formulate an algorithm for the utilization of Bayesian Support Vector Machines in a Multi-Task Learning framework. I modeled the generative process for classification,using Bayesian SVMs, of examples from different but related tasks
\end{enumerate}

\subsubsection*{Awards and achievements}

\renewcommand{\labelitemi}{$\textendash$}
\begin{itemize}
    \item Awarded the KVPY (Kishore Vaigyanik Protsahan Yojna) Scholarship in 2014.
    \item Secures an all India Rank of 589 in Joint Entrance Exam Advanced 2014 amongst 150,000 shortlisted candidates
\end{itemize}

\subsubsection*{Hobbies}
\renewcommand{\labelitemi}{$\textendash$}
\begin{itemize}
    \item I appreciate the arts of redditing and watching American Late Night Television
    \item I can also be found indulging in the occasional games of Cricket, Badminton and Poker 
\end{itemize}
\end{document}