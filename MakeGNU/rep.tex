\documentclass[]{article}

\usepackage{float}
\usepackage{graphicx}

\begin{document}

\begin{figure}
  \centering
  \includegraphics[width=\linewidth]{scatplot_1.eps}
  \caption{The number of Threads = 1 with a scatter plot of the execution time taken by different tasks having different number of elements}
\end{figure}

\begin{figure}
  \includegraphics[width=\linewidth]{scatplot_2.eps}
  \caption{The number of Threads = 2 with a scatter plot of the execution time taken by different tasks having different number of elements}
\end{figure}
\begin{figure}
  \includegraphics[width=\linewidth]{scatplot_4.eps}
  \caption{The number of Threads = 4 with a scatter plot of the execution time taken by different tasks having different number of elements}
\end{figure}

\begin{figure}
  \includegraphics[width=\linewidth]{scatplot_8.eps}
  \caption{The number of Threads = 8 with a scatter plot of the execution time taken by different tasks having different number of elements}
\end{figure}

\begin{figure}
  \includegraphics[width=\linewidth]{scatplot_16.eps}
  \caption{The number of Threads = 16 with a scatter plot of the execution time taken by different tasks having different number of elements}
\end{figure}

\begin{figure}
  \includegraphics[width=\linewidth]{linplot.eps}
  \caption{Line plot of different average execution times with respect to the number of elements plotted separately considering the number of active threads}
\end{figure}

\begin{figure}
  \includegraphics[width=\linewidth]{histplot.eps}
  \caption{Bar Plot of the average speed up given by increasing the number of threads and its variation with different number of elements}
\end{figure}

\begin{figure}
  \includegraphics[width=\linewidth]{hist_errorbar.eps}
  \caption{Bar Plot of the average speed up given by increasing the number of threads and its variation with different number of elements along with the variance in the ratio shown as an errorbar}
\end{figure}


\end{document}
