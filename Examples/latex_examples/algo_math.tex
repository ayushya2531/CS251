%\documentclass[a4paper, 10pt]{article}
\documentclass[a4paper, 10pt,twocolumn]{article}

\usepackage{amsmath}
\usepackage{lipsum}
%\usepackage{algorithm2e}
\usepackage{algorithm}
\usepackage{algpseudocode}


\title{Latex Algorithm and Equations}
\author{Debadatta Mishra}
\date{}
\begin{document}
\maketitle
\begin{abstract}
\lipsum[15]
\end{abstract}    
\section{Introduction}
\lipsum
\section{Algorithm}

%\begin{algorithm}
%\SetAlgoLined
%\KwIn{$A[0..N-1]$: Array of integers}
%\KwOut{$A[0..N-1]$: Sorted array of integers}
%$i \leftarrow$ 1 \;
%\While {$i <  N$}{
%    $j \leftarrow i$ \;
%    Iterate to find the place for this element in the sorted array \;
%   \While {$j > 0$ and $A[j-1] > A[j]$}{
%         swap($A[j]$, $A[j-1]$)\;
%         $j \leftarrow j -1$ \;
%   }
%   $i \leftarrow i + 1$ \;
%   
%    
%}
%
%\label{algo:ins_sort}
%\caption{Sort an array using insertion sort}
%\end{algorithm}

An example of writing algorithms is shown above.
Another way to write algorithms follow.

\begin{algorithm}
  \caption{Another way to write insertion sort}
  \label{algo:ins_sort1}
  \begin{algorithmic}[1]
     \Procedure{InsertionSort}{A[0..N]}\newline
     \Comment{A is an array of N integers}
      \State $i \leftarrow$ 1
      \While {($i <  N$)}
          \State $j \leftarrow i$ \newline
           \Comment{Iterate to find the place for this element in the sorted array}
         \While {($j > 0$ and $A[j-1] > A[j]$)}
            \State \textit{swap}($A[j]$, $A[j-1]$)
            \State $j \leftarrow j -1$ 
         \EndWhile
      \State $i \leftarrow i + 1$
      \EndWhile
     \EndProcedure 
  \end{algorithmic}
\end{algorithm}

\section{Math and Equations}
Some basic trigonometric identities are as follows,
\begin{equation}
\sin^2\theta + \cos^2\theta = 1
\end{equation}

\begin{equation*}
\sin(\alpha + \theta) = \sin \alpha \cos \theta + \cos \alpha \sin \theta
\end{equation*}

\indent
\begin{align}
\sin(\frac{\pi}{2} -\theta) &= \cos\theta \\
1 + \tan^2 \theta &= \sec^2 \theta\\
1 + \cot^2 \theta &= \csc^2 \theta
\end{align}

Now let us write some equations from calculus.
\begin{equation*}
\lim_{x \to \infty} \frac{\sin x}{x} = 1
\end{equation*}

\begin{align*}
\sum_{i=1}^n i^2 &= \frac{1}{2} n (n+1) \\
\int_0^{\frac{\pi}{2}} \cos\theta\,d\theta &= \sin\theta\Big|_0^\frac{\pi}{2} \\
  &= \sin\frac{\pi}{2} - \sin 0 \\
  &= 1 \\
\end{align*}
If  $f(x) = x^2 + 2x + 1$, \\ 
Then  $\frac{df(x)}{dx} = 2x + 2$ 
\section{Conclusion}
\lipsum[30]
\begin{itemize}
\item Item 1 is here
\item Item 2 is here
\item Item 3 is here
\end{itemize}

\begin{itemize}
\item[{\bf CS251:}] Is this course really useful? Some students consider this course to
                    be heavy considering it is just a 6 credit course. Hang on! You will 
                    get another chance to redeem your selves.
\item[{\bf CS330:}] Operating Systems is a old concept, but still we are required to study it! 
                    Concepts like virtualization---Xen~\cite{xen}, VMWare~\cite{vmware}--- are
                    making OS concepts relevant again. Have you heard about some advanced OS features like 
                    application containers like LXC~\cite{lxc} and Docker~\cite{docker}?   
\end{itemize}


\begin{thebibliography}{25}
\bibitem{xen}
Paul Barham, Boris Dragovic, Keir Fraser, Steven Hand, Tim Harris, Alex Ho, Rolf Neugebauer, Ian Pratt, and Andrew Warfield
\textit{Xen and the art of virtualization}
In Proceedings of the nineteenth ACM symposium on Operating systems principles (SOSP), 2003

\bibitem{vmware}
Carl A. Waldspurger
\textit{Memory resource management in VMware ESX server}
In Proceedings of USENIX OSDI, 2002


\bibitem{lxc}
\textit{Linux Containers: Infrastructure for container projects}
https://linuxcontainers.org/


\bibitem{docker}
\textit{Docker}
https://www.docker.com/

\end{thebibliography}
\end{document}
